\documentclass[a4paper,11pt]{article}
\usepackage[T1]{fontenc}
\usepackage[utf8]{inputenc}
\usepackage{lmodern}
\usepackage{graphicx}
\usepackage{amsmath}
\usepackage{amsfonts}
\usepackage{amssymb}
\usepackage{mathtools}

\DeclareMathOperator{\given}{\mid}

\begin{document}

\section*{HW2}

\begin{tabular*}{0.9\textwidth}{@{\extracolsep{\fill} } lll}
Jimmy Hold\"{o} & & Jared Karr\\
890130-6319 & & 801120-4693\\
\it{gusholji@student.gu.se} & & \it{karr@student.chalmers.se}\\
\end{tabular*}

\section{Theoretical problems}
\subsection{Bayes classifer}
We observe the following probabilities:
\begin{align*}
&&P(c=1)&=1/2 & &\\
P(x_1=0)&=1/2 &&& P(x_1=0\given c=1)&=1/4\\
P(x_2=1)&=3/8 &&& P(x_2=1\given c=1)&=1/2\\
P(x_3=1)&=1/2 &&& P(x_3=1\given c=1)&=3/4
\end{align*}

\paragraph{(a)}
\begin{align*}
  P(c=1\given x=(0, 1, 1))
  &=\frac{
    P(x_1=0\given c=1)
    P(x_2=1\given c=1)
    P(x_3=1\given c=1)
    P(c=1)
  }{
    P(x_1=0)
    P(x_2=1)
    P(x_3=1)
  }\\
  &=\frac{
    (1/4)(1/2)(3/4)(1/2)
  }{
    (1/2)(3/8)(1/2)
  }=1/2
\end{align*}

\paragraph{(b)}
\begin{align*}
  P(c=1\given x=(0, 1, 1))
  &=\frac{
    P(x_1=0\given c=1)
    P(x_2=1\given c=1)
    P(c=1)
  }{
    P(x_1=0)
    P(x_2=1)
  }\\
  &=\frac{
    (1/4)(1/2)(1/2)
  }{
    (1/2)(3/8)
  }=1/3
\end{align*}

\subsection{Extending na\"ive Bayes}
\section{Practical problems}
\subsection{}

\subsection{}
\end{document}
