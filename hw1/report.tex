\documentclass[a4paper,11pt]{article}
\usepackage[T1]{fontenc}
\usepackage[utf8]{inputenc}
\usepackage{lmodern}
\usepackage{graphicx}
\usepackage{amsmath}
\usepackage{amsfonts}
\usepackage{amssymb}
\usepackage{mathtools}
\usepackage{epstopdf}

\DeclareMathOperator{\given}{\mid}

\begin{document}

\section*{HW1}

\begin{tabular*}{0.9\textwidth}{@{\extracolsep{\fill} } lll}
Jimmy Hold\"{o} & & Jared Karr\\
890130-6319 & & 801120-4693\\
\it{gusholji@student.gu.se} & & \it{karr@student.chalmers.se}\\
\end{tabular*}

\section{Theoretical problems}
\subsection{Maximum likelihood estimator (MLE)}
The likelihood of a set of observations $\mathbf{x_1},\dots,\mathbf{x_n}\in\mathbb{R}^p$ drawn I.I.D. from a $p$-dimensional spherical Gaussian distribution having mean $\boldsymbol\mu\in\mathbb{R}^p$ and variance $\sigma^2\in\mathbb{R}_{>0}$ is
\begin{align*}
  P(\mathbf{x_1},\dots,\mathbf{x_n}\given \boldsymbol\mu,\sigma^2)
    &=\prod_{i=1}^n\mathcal{N}(\mathbf{x_i}\given \boldsymbol\mu, \sigma^2)\\
    &=\prod_{i=1}^n\left(
        \frac{1}{\sqrt{2\pi}^p\sigma}
      \right)\exp\left(
        \frac{-(\mathbf{x_i}-\boldsymbol\mu)^\top(\mathbf{x_i}-\boldsymbol\mu)}
             {2\sigma^2}
      \right).
\end{align*}
To find $\sigma_\textrm{MLE}$, it is easier to maximize the log-likelihood
\begin{equation*}
  \log P(\mathbf{x_1},\dots,\mathbf{x_n}\given\boldsymbol\mu,\sigma^2)
    =\sum_{i=1}^n
      -p\log\sqrt{2\pi}
      -\log\sigma
      -\frac{(\mathbf{x_i}-\boldsymbol\mu)^\top(\mathbf{x_i}-\boldsymbol\mu)}{2\sigma^2} 
\end{equation*}
with respect to $\sigma$. Taking the appropriate partial derivative and setting to zero,
\begin{align*}
\frac{\partial\log P}{\partial\sigma}
  &=\sum_{i=1}^n
    -\frac{1}{\sigma}
    +\frac{1}{\sigma^3}\cdot(\mathbf{x_i}-\boldsymbol\mu)^\top(\mathbf{x_i}-\boldsymbol\mu)\\
  &=-\frac{n}{\sigma}
    +\frac{1}{\sigma^3}\sum_{i=1}^n(\mathbf{x_i}-\boldsymbol\mu)^\top(\mathbf{x_i}-\boldsymbol\mu)=0\\
n\sigma^2
  &=\sum_{i=1}^n(\mathbf{x_i}-\boldsymbol\mu)^\top(\mathbf{x_i}-\boldsymbol\mu)\\
\sigma_\textrm{MLE}&=\sqrt{\frac{1}{n}\sum_{i=1}^n(\mathbf{x_i}-\boldsymbol\mu_\textrm{MLE})^\top(\mathbf{x_i}-\boldsymbol\mu_\textrm{MLE})}.
\end{align*}
\subsection{Posterior distributions}
\section{Practical problems}
\subsection{Spherical Gaussian estimation}
\paragraph{(a)}
\begin{align*}
\boldsymbol\mu_\textrm{MLE}&=\frac{1}{n}\sum_{i=1}^n\mathbf{x_i}\\
\sigma_\textrm{MLE}&=\sqrt{\frac{1}{n}\sum_{i=1}^n(\mathbf{x_i}-\boldsymbol\mu_\textrm{MLE})^\top(\mathbf{x_i}-\boldsymbol\mu_\textrm{MLE})}.
\end{align*}
\subsection{MAP estimation}
\end{document}
